\chapter{More on geometric algebra}

Although vector geometric algebra provides a simple, convenient model for embedding our
understanding of vectors into the framework of geometric algebra, it is by no means the only model.
Models such as \textit{projective geometric algebra} (PGA) and \textit{conformal geometric algebra}
(CGA) use extra dimensions to assist in the modeling of various kinds of objects in space.

\section{Familiar number systems are geometric algebras}

We've previously seen how the 2D vector geometric algebra contains the complex numbers in its even
subalgebra. The complex numbers themselves are a geometric algebra, $Cl_{0,1}(\mathbb{R})$. Here,
we've extended the Clifford algebra symbol to include the number of dimensions that square to a
negative number - that dimension is the imaginary numbers, with the real numbers being part of the
scalar field that always squares to 1.

Similarly, the quaternions also comprise a geometric algebra, $Cl_{0,2}(\mathbb{R})$. It's easy to 
show that the two vectors and the bivector of the space map onto the quaternionic elements $i$, $j$,
and $k$ (the order of correspondence is irrelevant).

Unfortunately, we can't go higher. The octonions aren't associative, violating a fundamental
requirement that Clifford algebras hold - they are unital associative algebras. The Cayley-Dickson
contstruction allows for the construction of algebras over the field of real numbers with
increasingly loose constraints as the number of dimensions increases.

\section{Projective geometric algebra}

A \textit{projective geometric algebra} (PGA) is a geometric algebra that contains one or more extra
projective dimensions. These dimensions are \textit{projective}, meaning that they are essentially
flattened into the remaining dimensions. When we calculate a length, components of a multivectors
that contain the projective dimensions square to zero.

We will extend our descriptions of the underlying Clifford algebra by adding a third dimension index
representing the components that square to zero. The n-dimensional projective geometric algebra is
modeled by the Clifford algebra $Cl_{n,0,1}(\mathbb{R})$.

Projective geometric algebra provides a framework for treating translations and rotations
equivalently. This makes it an extremely useful model for working with space groups - the symmetries
that are possible in crystals. It also serves as a foundation for 3D computer graphics - the
projective properties naturally describe the process of flattening a 3D object into a 2D
representation.

\section{Conformal geometric algebra}
