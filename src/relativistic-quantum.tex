\chapter{Relativistic quantum mechanics}

In the previous on special relativity, we saw how placing space and time on equal footing provides a
simpler framework for working with seemingly complicated phenomena, such as electromagnetism. We
were able to identify the electric and magnetic field vectors with the spacelike and timelike
bivectors of the spacetime algebra, and unified our descriptions of the fields into the single
equation $\nabla F = J$.

Now we can extend this to our understanding of quantum mechanics. Looking back at the Schrödinger
equation, it becomes obvious that there is a problem with the formulation that makes it unfitting
for extension: the time derivative is first-order, but the spatial derivatives are second-order.
Time and space are clearly not being treated equally.

\section{The Klein-Gordon equation}

When Erwin Schrödinger derived his namesake equation, he began with a relativistic equation, but
eschewed the relativistic setting. This equation is the \textit{Klein-Gordon equation}. We'll follow
this derivation

The energy in special relativity is the sum of two components: the mass-energy and the kinetic
energy. The expression for this energy is

$$E^2 = (pc)^2 + (mc^2)^2$$

We can substitute our defnitions for the energy operator and momentum operator that we used earlier
on when exploring non-relativistic quantum mechanics.

\begin{align}
    \hat{E} = i\hbar\frac{\partial}{\partial t}   &   \hat{p} = -i\hbar\nabla
\end{align}

As with the Schrödinger equation, we'll have these operators act on a wavefunction $\psi$:
$$
\left(i\hbar\frac{\partial}{\partial t}\right)^2 \psi =
\left(-i\hbar c \nabla\right)^2 \psi +
\left(mc^2\right)^2 \psi
$$
However, we can simplify this equation a bit. We'll move the second derivative terms to one side:
$$
\left(i\hbar\frac{\partial}{\partial t}\right)^2 \psi -
\left(-i\hbar c \nabla\right)^2 \psi =
\left(mc^2\right)^2 \psi
$$
Now we can resolve the squared differential operators. We'll use the equality $(i\hbar)^2 =
(-i\hbar)^2 = -\hbar^2$, and combine the double derivatives into a single linear operation acting on
$\psi$.
$$
-\hbar^2 \left(\frac{\partial^2}{\partial t^2} - c^2 \nabla^2\right) \psi = \left(mc^2\right)^2 \psi
$$
Now we'll divide both sides of this equation by $-c^2 \hbar^2$:
$$
\left(\frac{1}{c^2}\frac{\partial^2}{\partial t^2} - \nabla^2\right) \psi =
-\left(\frac{mc}{\hbar}\right)^2 \psi
$$
Because we're working with the east coast metric, we'll multiply everything by $-1$ to make sure
that the spatial derivative components are positive and the temporal component is negative:
$$
\left(\nabla^2 - \frac{1}{c^2}\frac{\partial^2}{\partial t^2}\right) \psi =
\left(\frac{mc}{\hbar}\right)^2 \psi
$$
Finally, we can define the differential operator $\partial^2 = \nabla^2 -
\frac{1}{c^2}\frac{\partial^2}{\partial t^2}$ and substitute that in:
$$
\partial^2 \psi = \left(\frac{mc}{\hbar}\right)^2 \psi
$$

\subsection{Problems with the Klein-Gordon equation}

The problem with the Klein-Gordon equation is that it is a second order equation in spacetime.
Although it is possible to use the equation

\section{The Dirac equation}

The problems with the Klein-Gordon equation prevented it from being used directly in experimental
and computational study. Physicists at the time sought a first-order equation. Although it seemed
relatively straightforward to take the square root of the energy...
$$E^2 = \sqrt{(pc)^2 + (mc^2)^2}$$
...the problem of retrieving a first-order operator such by taking the square root of $\partial^2$
seemed somewhat hopeless. Although it is in principle possible to calculate the square root by
performing a Taylor expansion of the square root function and evaluating it for $\partial^2$, it
results in a messy expression that may not be practical to use.

Paul Dirac was one of the physicists pondering this problem, and he had a different idea: instead of
performing any sort of infinite expansion of $\partial^2$, he'd work around this problem by
factoring it into a more tractable form:
$$
\partial^2 = \left(\gamma^0 \frac{\partial}{\partial t} +
\gamma^1 \frac{\partial}{\partial x} +
\gamma^2 \frac{\partial}{\partial y} +
\gamma^3 \frac{\partial}{\partial z}
\right)^2
$$
where $\gamma^\mu$ are some coefficients to be determined. The issue now becomes making cross terms,
such as $\gamma_0 \frac{\partial}{\partial t} \gamma_3 \frac{\partial}{\partial z}$ disappear, as
they aren't present in the definition of $\partial^2$. This implies two sets of conditions: first,
$\gamma^\mu \gamma^\nu = -\gamma^\nu \gamma^\mu$ if $\mu \neq \nu$. Second, $(\gamma^\mu)^2 = -1$
when $\mu = 0$, and $(\gamma^\mu)^2 = 1$ when $\mu > 0$.

Dirac realized that no ordinary numbers have this property, but it would be possible to construct a
set of matrices that met this requirement. Those matrices are known as the \textit{gamma matrices},
and they are a set of four $4\times4$ real matrices. These were described as operating on a
4-dimensional complex vector $\psi$, whose components described left-handed and right-handed
electrons and then-undiscovered positrons, which had the opposite charge of electrons. Although this
was poorly understood initially, the discovery of the positron validated Dirac's theory.

Unknown to Dirac at the time, the conditions that define the gamma matrices are \textit{exactly
those that define the basis vectors of the spacetime algebra}! This opens the interpretation of the
wavefunction up: rather than trying to make sense of the individual components of the wavefunction's
complex vector representation, we treat the wavefunction as a spinor field, where at every point in
space there is a multivector value consisting of sums of even basis blades.

By representing the partial derivatives in an indexed notation ($t = x^0, x = x^1, y = x^2,
z = x^3$), we can write the Dirac equation in a commonly seen form:
$$
\gamma^\mu \frac{\partial}{\partial x^\mu} \psi = \frac{mc}{\hbar} \psi
$$
More commonly, the components are subtracted so one side of the equation is zero:
$$
\left(\gamma^\mu \frac{\partial}{\partial x^\mu} - \frac{mc}{\hbar}\right) \psi = 0
$$
And in natural units ($\hbar = c = 1$):
$$\
\left(\gamma^\mu \frac{\partial}{\partial x^\mu} - m\right) \psi = 0
$$

\subsection*{What about the imaginary unit?}

In many resources, the Dirac equation (in natural units) has an extra imaginary factor attached to
it:
$$\
\left(i\gamma^\mu \frac{\partial}{\partial x^\mu} - m\right) \psi = 0
$$
However, our derivation managed to completely bypass the imaginary unit. How did this happen?

In many derivations of the Dirac equation, the $\partial^2$ operator is factored with an included
imaginary unit to account for the negative sign of one set of dimensions:
$$
\partial^2 = \left(i\gamma^0 \frac{\partial}{\partial t} +
\gamma^1 \frac{\partial}{\partial x} +
\gamma^2 \frac{\partial}{\partial y} +
\gamma^3 \frac{\partial}{\partial z}
\right)^2
$$
This allows for more consistent conditions the $\gamma^\mu$ factors must meet: in particular, the
second criterion simplifies to $\left(\gamma^\mu\right)^2 = 1$. In order to complete the derivation,
the $\gamma^1$, $\gamma^2$, and $\gamma^3$ matrices are replaced $i\gamma^1$, $i\gamma^2$, and
$i\gamma^3$. However, this introduction of an imaginary unit is completely unnecessary if the first
derivation is followed, and the dynamics of the spinor field are the same.
