\chapter{Special relativity}

\section{The speed of light}

The first evidence that light travels at a finite speed was found in 1676 by the Danish astronomer
Ole Rømer. He noticed discrepancies in the timing of the eclipses of the Jovian moon Io by Jupiter
that depended on the phases of the orbits of Earth and Jupiter. Although his observations were
initially controversial, after some time, further experiments verified his results and refined the
measured value of the speed of light.

However, the more interesting question was whether light travels through some sort of medium (a
\textit{luminiferous aether}) that provides a preferred frame of reference. Like light, sound
travels at a fixed speed, but that speed is only constant in a frame that holds the medium
stationary. The Michelson-Morley experiment aimed to determine whether this is the case for light.
By exploiting phenomena that would, in principle, alter the velocity of the experiment relative to
the luminiferous aether, such as the Earth's rotation or progression along its orbit, they tested
if the measured speed of light varied proportionally to those phenomena. They found no variance
within experimental error.

The observations from this experiment were surprising - it showed that light really does travel at
the same speed regardless of the source's position or velocity. This had some strange implications.
Consider Albert Einstein's \textit{Gedankenexperiment} (thought experiment) involving a stationary
clock and a high speed train. In this scenario, a high speed train passes through a train station
with a clock visible to a passenger. As the train approaches the station, the passenger the light
emitted by the clock, and that light appears to be traveling at a fixed speed. However, the train
passenger will intercept the visible changes in clock state faster than the clock ticks. Therefore,
the clock appears to run faster on the train than it does in the station! As the train passes and
departs, the light from the train will have to travel greater distances to catch up with the train,
which causes the clock to appear to slow down for the train passenger! Similar thought experiments,
such as those involving photons bouncing between two mirrors, show that if the speed of light is
fixed, then the measurement of time depends on relative velocities.

Moreover, the measurement of space is relative, too! If we try to measure the length of the train
using its speed and the times at which the front and rear of the train pass by a certain marker at
the station, an observer on the platform will measure different times than a pair of observers
stationed at the front and rear of the train for the reasons mentioned above.
