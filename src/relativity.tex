\chapter{Special relativity}

\section{Background}

\subsection{The speed of light}

The first evidence that light travels at a finite speed was found in 1676 by the Danish astronomer
Ole Rømer. He noticed discrepancies in the timing of the eclipses of the Jovian moon Io by Jupiter
that depended on the phases of the orbits of Earth and Jupiter. Although his observations were
initially controversial, after some time, further experiments verified his results and refined the
measured value of the speed of light.

However, the more interesting question was whether light travels through some sort of medium (a
\textit{luminiferous aether}) that provides a preferred frame of reference. Like light, sound
travels at a fixed speed, but that speed is only constant in a frame that holds the medium
stationary. The Michelson-Morley experiment aimed to determine whether this is the case for light.
By exploiting phenomena that would, in principle, alter the velocity of the experiment relative to
the luminiferous aether, such as the Earth's rotation or progression along its orbit, they tested
if the measured speed of light varied proportionally to those phenomena. They found no variance
within experimental error.

The observations from this experiment were surprising - it showed that light really does travel at
the same speed regardless of the source's position or velocity. This had some strange implications.
Consider Albert Einstein's \textit{Gedankenexperiment} (thought experiment) involving a stationary
clock and a high speed train. In this scenario, a high speed train passes through a train station
with a clock visible to a passenger. As the train approaches the station, the passenger the light
emitted by the clock, and that light appears to be traveling at a fixed speed. However, the train
passenger will intercept the visible changes in clock state faster than the clock ticks. Therefore,
the clock appears to run faster on the train than it does in the station! As the train passes and
departs, the light from the train will have to travel greater distances to catch up with the train,
which causes the clock to appear to slow down for the train passenger! Similar thought experiments,
such as those involving photons bouncing between two mirrors, show that if the speed of light is
fixed, then the measurement of time depends on relative velocities.

Moreover, the measurement of space is relative, too! If we try to measure the length of the train
using its speed and the times at which the front and rear of the train pass by a certain marker at
the station, an observer on the platform will measure different times than a pair of observers
stationed at the front and rear of the train for the reasons mentioned above.

\subsection{The Minkowski metric}

The \textit{Minkowski metric}, $\eta_{\mu\nu}$ is the metric tensor of space as described in special
relativity. In an orthonormal basis:
$$
\eta_{\mu\nu} = \begin{bmatrix}
    -\frac{1}{c^2} & 0 & 0 & 0 \\
    0 & 1 & 0 & 0 \\
    0 & 0 & 1 & 0 \\
    0 & 0 & 0 & 1 \\
\end{bmatrix}
$$
The first row and column correspond to the time dimension, and the rest of rows and columns
correspond to the three spatial dimensions.

However, this is not the only way we can define the Minkowski metric. We used the \textit{East Coast
metric}, or the mostly positive metric. However, the metric of spactime is essentially unchanged if
we use a metric that has the spacelike dimensions square to negative values:
$$
\eta_{\mu\nu} = \begin{bmatrix}
    \frac{1}{c^2} & 0 & 0 & 0 \\
    0 & -1 & 0 & 0 \\
    0 & 0 & -1 & 0 \\
    0 & 0 & 0 & -1 \\
\end{bmatrix}
$$
These correspond to two different algebras, $\ClR{1,3}$ and $\ClR{3,1}$ respectively. Although these
algebras are similar in structure, it is critical to note that \textit{they are not isomorphic!}

Throughout this work, we will stick to the East Coast convention, and work in the Clifford algebra
$\ClR{3,1}$, which we consider the canonical spacetime algebra. Working with this convention allows
us to avoid introducing imaginary units in future equations.

\section{The spacetime algebra}

Previously, we used $\ClR{3}$, the algebra of physical space, to provide a set of basis elements - 
vectors, bivectors, and trivectors - that can describe physical phenomena. For Minkowski space, we
can extend the algebra to $\ClR{3,1}$ - the \textit{spacetime algebra} (STA).

\subsection{The even subalgebra and spacetime splits}

Like with other Clifford algebras, we will find it useful in many cases to restrict ourselves to the
even subalgebra of STA, which is just $\ClR{3}$, the algebra of physical space. Interestingly, both
$\ClR{3,1}$ and $\ClR{1,3}$ both contain $\ClR{3}$ as an even subalgebra, which may explain why both
conventions are commonly used (the West Coast metric for particle physicists, the East Coast metric
for relativists), even though the two are not isomorphic.

It is useful to investigate the elements of the even subalgebra of STA and understand how they are
linked to those of $\ClR{3}$. The even subalgebra contains the scalar $1$, the bivectors $e_0 e_1$,
$e_0 e_2$, $e_0 e_3$, $e_1 e_2$, $e_1 e_3$, $e_2 e_3$, and the pseudoscalar $i = e_0 e_1 e_2 e_3$.
Mapping the scalar and pseudoscalar is trivial, but we want to describe the subalgebra so that some
spacetime bivectors correspond to spatial vectors, and the rest correspond to bivectors. Thankfully,
the unique behavior of $e_0$ gives us a natural way to map the STA bivectors onto APS vectors and
bivectors. Our mapping just needs to drop the $e_0$ element:
$$
\begin{matrix}
    e_0 e_1 \ClR{3,1} \mapsto e_1 \ClR{3} & e_1 e_2 \ClR{3,1} \mapsto e_1 e_2 \ClR{3} \\
    e_0 e_2 \ClR{3,1} \mapsto e_2 \ClR{3} & e_1 e_3 \ClR{3,1} \mapsto e_1 e_3 \ClR{3} \\
    e_0 e_3 \ClR{3,1} \mapsto e_3 \ClR{3} & e_2 e_3 \ClR{3,1} \mapsto e_2 e_3 \ClR{3} \\
\end{matrix}
$$

A geometric interpretation of the even subalgebra is that it describes a moment in time in a special
relativistic framework. Since we can't see the timelike components of STA bivectors in that moment,
those bivectors become vectors. The other bivectors, which have only spacelike components, describe
oriented areas in space, and thus rotations. Sometimes, we use the notation $\sigma_n = e_0 e_n$ to
denote the basis of the even subalgebra - this highlights their equivalence to the Pauli matrices, a
matrix representation of APS, commonly abbreviated to $\sigma$.

We can project any STA element into the even subalgebra by multiplying it by $e_0$. This causes the
timelike component to collapse into a scalar, and the spacelike component to become a bivector. For
example, an event vector $x$ can be projected into the even subalgebra:
$$
e_0 x = x^0 + x^\mu e_0 e_\mu
$$
Whether we multiply on the left or right does not matter, provided we keep track of the sign. By
convention, we will multiply on the left so that all signs are positive when the basis bivectors use
lexicographic ordering:
$$
x e_0 = x^0 + x^\mu e_\mu e_0 =  x^0 - e_0 x^\mu e_1
$$
This process is known as a \textit{spacetime split}, and it can be used to rephrase a relativistic
framework into a non-relativistic one. While many theories are greatly simplified in a relativistic
setting, for practical calculations, such an operation is useful, especially for time-independent
phenomena - one practical example is the calculation of the energy of a chemical system at
equilibrium.
