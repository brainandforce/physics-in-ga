\chapter{Electromagnetism}

Electromagnetism is arguably the dominant fundamental force in our daily lives.

\section{Charges and the electric field}

Every object in space has some electrical charge associated with it. This charge can be either
positive or negative. Exactly what we assign "positive" or "negative" to is irrelevant.

\subsection{Coulomb's law}

Coulomb's law describes the force that two point charges $q_1$ and $q_2$ experience. The
electromagnetic force drops off over distance, so we'll want to keep track of the displacement 
vector $\vec{r}$ between them. From experimental observation, we find that the electromagnetic
force decreases with the square of the distance. This motivates a form of the law which contains
some term dependent on the inverse of $\vec{r}^2$:

$$\left|F\right| \propto \frac{\left|q_1\right|\left|q_2\right|}{\hat{r}^2}$$

We'll introduce a constant $k$, known as \textit{Coulomb's constant}, which allows us to convert
units of charge squared over distance squared to a force. The value of Coulomb's constant is $k =
8.9875517923(14) \times 10^9 \; \mathrm{ N \cdot m^2 \cdot C^{-2}}$.

Now we have a law that tells us the magnitude of the force - but we want the force as a vector, not
just a scalar. To do this, we'll need to think about how charges interact. The attracting case, 
with opposite charges, gives us a hint: when the charges multiply to a negative number, the force
points in the same direction as the displacement vector. That means that we should prefix the law
with a negative sign to get the right behavior.

Now we need to account for the direction that the force points. The force is constrained to lie 
along the displacement vector between the charges. However, we've already accounted for the 
inverse-square behavior of electric force, so we need to include a factor of the 
\textit{normalized} vector along that direction: $\hat{r} = \frac{\vec{r}}{\sqrt{\vec{r}^2}}$.

$$\hat{F} = -k\frac{q_1 q_2}{\vec{r}^2}\frac{\vec{r}}{\sqrt{\vec{r}^2}}
 = -k\frac{q_1 q_2}{\vec{r}^2}\hat{r}$$

\subsection{The electric field}

\subsection{The electric potential}

\subsection{Gauss's law}

\section{Maxwell's equation}

If you have studied electromagnetism before, you likely remember Maxwell's equations - at least by
name. The conventional description of the equation is

$$\nabla \cdot E = \frac{\rho}{\epsilon_0}$$
$$\nabla \cdot B = 0$$
$$\nabla \times E = -\frac{\partial B}{\partial t}$$
$$\nabla \times B = \mu_0 J - \frac{1}{c^2} \frac{\partial E}{\partial t}$$

We've recovered these laws through the 

Now we can assemble \textit{Maxwell's equation}: $$\nabla F = J$$

This single equation describes the entirety of the electromagnetic force

\subsection{What about potentials?}



\section{Making it relativistic}


\section{Modifications to electromagnetism}

\subsection{Higher dimensions}
