\chapter{Classical mechanics}

The standard introduction to classical mechanics

\section{Action}

\subsection{The best possible path}

\textit{Candide, ou l'Optimisme} by Voltaire is a scathing satire of Leibniz's philosophy of 
optimism: in summary, it states that we live in the best of all possible worlds. This idea is,
frankly, nonsensical, given the horrible state of much of the world we live in. Therefore, we will
use this idea to develop a new, highly generalizable understanding of mechanics.

Consider the trajectory of a thrown object. If you already have somewhat of a background in
mathematics or physics, you might know that, barring effects such as air resistance, the object
moves along a parabolic trajectory. Using the nonsensical idea mentioned previously, we will assert
that this trajectory - or the trajectory of any classical object - is the best of all possible
trajectories it may take.

Although "best" is a subjective term, what we can do is build a mathematical machine that takes the
trajectory of an object and returns some value of "goodness". To follow convention, from here on,
we'll refer to this "goodness" as the \textit{action}, which is given the symbol $S$. The
\textit{action functional}, $S\left[q\right]$takes a trajectory $q\left(t...\right)$, which depends
on time and possibly other parameters, and returns the action, which is a real number. Now we can
use calculus to make a statement about the value of the action for the "best" trajectory. Because
any change to the trajectory would result in a new trajectory that is not the "best", we know that
the change in action over an infinitesimal change in the trajectory must be zero:

$$\frac{\delta S}{\delta q} = 0$$

The new $\delta$ symbol describes a \textit{functional derivative}, which describes the change in a
functional with respect to a corresponding change in the input function. This principle is known as
the \textit{stationary action principle}.

For those of you who remember how optimization works in calculus, you may recall the second
derivative test: just because a function has a point where its derivative is zero does not mean
that the point is a maximum. It could be a minimum, or perhaps a more complicated terrace point.
The value of the second derivative determines what the point actually is. However, in the case of
action, we don't actually need to perform this test: regardless of the nature of the stationary
point, we can still get the right dynamics from ensuring $\frac{\delta S}{\delta q} = 0$. (This
does mean that we could actually be looking at the \textit{worst} of all possible trajectories, if
we still think of the action as a measure of the goodness of a trajectory!)

Sometimes, the term \textit{least action principle} is used instead, but for the reason outlined
above, it's a little misleading. We'll stick to calling it the stationary action principle.

\subsection{Hamilton's principal function}

Although we've formally defined the action as a functional of the trajectory $q\left(t...\right)$,
we can think of the action a little differently. For a physical situation, some action function
$S\left(t_0..., t...\right)$ can tell us everything about the system's dynamics, provided we know
the state of the object at two points in time. This function is referred to as \textit{Hamilton's
principal function}. 

The challenge now is determining how to adequate describe a state. We can get an idea of this from
our own intuition. If we want to change the trajectory of a falling object, we can make two types
of changes: either we could drop it from a different position, or we could give it a different
starting velocity. Combined with the time, we can use position and velocity to completely describe
a state which may be used as an argument for Hamilton's principal function.

Although it might seem that we need to know two points that are along the trajectory the object
takes, this is actually not the case. In order to find the dynamics of the system, we can compare
a known state $(x,v,t)$ and compare it to an arbitrary state $(x_0,v_0,t_0)$, giving a value for
the action. To generate the rest of the trajectory, we apply the stationary action principle:

$$\frac{d}{dt}S\left(x_0, v_0, t_0, x, v, t\right) = 0$$

We can think of the total derivative $\frac{dS}{dt}$ as the sum of several partial derivatives:

$$
\frac{dS}{dt} = 
\frac{\partial S}{\partial x_0} \frac{dx_0}{dt} + 
\frac{\partial S}{\partial v_0} \frac{dv_0}{dt} + 
\frac{\partial S}{\partial t_0} \frac{dt_0}{dt} + 
\frac{\partial S}{\partial x} \frac{dx}{dt} + 
\frac{\partial S}{\partial v} \frac{dv}{dt} + 
\frac{\partial S}{\partial t} \frac{dt}{dt} = 0
$$

Since the arbitrary initial state doesn't depend on the time, we can simplify this equation
significantly:

$$
\frac{dS}{dt} = 
\frac{\partial S}{\partial x} \frac{dx}{dt} + 
\frac{\partial S}{\partial v} \frac{dv}{dt} + 
\frac{\partial S}{\partial t} = 0
$$

We can perform one final rearrangement to get a simpler for of the equation we'll work with:

$$
\frac{\partial S}{\partial x} \frac{dx}{dt} + 
\frac{\partial S}{\partial v} \frac{dv}{dt} =
-\frac{\partial S}{\partial t}
$$

\subsection{Symmetries, conservation laws, and Noether's theorem}

In order to move further, we'll have to exploit the symmetries of the systems we're looking at. A
\textit{symmetry} is an operation that leaves a system unchanged when applied to the system. For a
familiar example, consider what happens when a square is rotated 90 degrees - unless the square is
marked in some way, it won't appear to have rotated at all. 90 degree rotations are thus symmetries
of the square.

Let's think about what symmetries occur in physical systems. Here, we'll use the example of two
masses connected by a massless spring.In this system, there is an equilibrium distance $r_0$ where
the spring exerts no force on the masses, but a force is applied that is proportional to the
distance $r - r_0$ the object lies from that equilibrium point. If we change the distance between
the masses by some value  $\Delta r$, this affects the dynamics of the system. But if we apply that
change to all length parameters in the entire system, then the dynamics of the system won't change.
Similarly, stationary action holds for those trajectories if we globally change time by another
quantity $\Delta t$. In general, we can think of these changes as moving the experimental apparatus
to somewhere new in space (giving the spatial translation symmetry) or performing the experiment at
a different time (giving the time translation symmetry). 

In both cases, before and after transformation, the stationary action principle holds:

$$\frac{\delta S}{\delta q} = \frac{\delta S}{\delta q'} = 0$$

It's critical to note that these symmetries may change $S$ by some constant $\Delta S$, but that
change is irrelevant when taking the time derivative.

This is a bit of a strange path to take, but it'll come in handy shortly. What we're going to do
is look at what happens to the action when we not only shift a symmetric coordinate, but vary that
shift over time. In mathematical terms, we're going to perform the transformation 
$q \rightarrow q + \varepsilon\left(t\right)\Delta q$, then determine how this affects the action.


\section{Lagrangian mechanics}


